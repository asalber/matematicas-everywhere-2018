\documentclass[a4paper,10pt,twoside]{article}

\setlength{\parskip}{2mm}


\usepackage[utf8]{inputenc}
\usepackage{amsfonts}
\usepackage{amsmath}
\usepackage{mathpazo}
\usepackage{multicol}
\usepackage{amssymb}
\usepackage{amsthm}
\usepackage{setspace}
\usepackage{fancybox}
\usepackage[spanish]{babel}
\usepackage[T1]{fontenc}
%\usepackage[pdftex]{graphicx} %Utilizarlo sólo si voy a compilar con pdftex y tengo imágenes pixeladas (no vectoriales) con las gráficas.
\usepackage{cancel}
\usepackage{fancyhdr}
\usepackage{eurosym}
\usepackage[dvips]{graphicx}%Utilizarlo para  cargar archivo de extensión *.eps (gráficos vectoriales)en el documento
\DeclareGraphicsExtensions{.png,.pdf,.jpg}
\usepackage{pstricks-add}%Utilizarlo para cargar código ps-tricks de geogebra por ejemplo (no usar con \usepackage[pdftex]{graphicx})
\usepackage[colorlinks=true,linkcolor=blue]{hyperref}
\usepackage{hyperref}
\hypersetup{
    colorlinks,%
    citecolor=black,%
    filecolor=black,%
    linkcolor=black,%
    urlcolor=black
}
\usepackage{wrapfig}
\usepackage{textcomp}
\usepackage{nonfloat}%para poder poner notas al pie en los pies de Figuras o de Tablas
                     %\begin{minipage}{\textwidth}
                     %   \centering
                     %   \begin{tabular}{|c|}
                     %      \hline  Tabular stuff here \\ \hline
                     %   \end{tabular}
                     %   \tabcaption[Text of the caption (second example)]{Text of the caption (second example)\footnotemark}
                     % \end{minipage}
                     % \\[\intextsep]
                     %\footnotetext{Text of the footnote (second example)}
\usepackage{colortbl}%Para colores en tablas
\usepackage{verbatim} %Para utilizar comentarios comentarios
\usepackage{fixltx2e}% superíndices y subindices (\textsubscript{} y \textsuperscript{}) en modo texto

%Nueva geometría de la página, con respecto a la de por defecto
\usepackage[text={14.7cm,22.6cm},centering]{geometry}
\addtolength{\headsep}{-0.35cm}
\addtolength{\footskip}{-0.3cm}
\addtolength{\voffset}{0.6cm}%{-0.65cm}
\addtolength{\hoffset}{-0.05cm}%{-0.65cm}


%%%%% Para los encabezados y pies de Página
\pagestyle{fancy}
\lhead[\scriptsize{\textit{\href{mailto:mailautor@xxxx.es}{Nombre y Apellidos del/los Autor/es}}}]{\scriptsize{\textit{Título del Artículo}}} % a completar por equipo editorial
\chead{}
\rhead[\scriptsize{\textit{Título del Artículo}}]{\scriptsize{\textit{\href{mailto:mailautor@xxxx.es}{Nombre y Apellidos del/los Autor/es}}}} % a completar por equipo editorial
\lfoot[\thepage \ \  |  \ \scriptsize{\textit{5\textsuperscript{as} Jornadas ``Matemáticas Everywhere''}}]{\scriptsize{\textit{XX y XX de XXXXX de 2018, Castro Urdiales}}}% a completar por equipo editorial
\cfoot{}
\rfoot[\scriptsize{\textit{XX y XX de XXXXX de 2018, Castro Urdiales}}]{\scriptsize{\textit{5\textsuperscript{as} Jornadas ``Matemáticas Everywhere''}}\ \ \ \normalsize{|} \   \thepage}
%%%%% Fin de encabezados y pies de página

%%%%% Genera los pies de Figuras y Tablas en itálicas y a tamaño footnote
\usepackage[font={footnotesize,it}, format=plain, labelformat=simple, labelsep=period, center]{caption}
\renewcommand\thefigure{\arabic{figure}} % Genera numeración X
\renewcommand\thetable{\arabic{table}} % Genera numeración X

\setcounter{page}{1}%para comenzar la numeración de las páginas en documentos. Dejar 1 por defecto.



\begin{document}
%Cambiar Palabra Cuadros por Tablas
\renewcommand{\tablename}{Tabla} 

\title{\vspace{-8mm}5\textsuperscript{as} Jornadas ``Matemáticas Everywhere''\\\vspace{4mm}
Título del Artículo en castellano\\\vspace{4mm}
Title of the Article in english\vspace{-3mm}}  %Para poner el título del artículo
\author{\href{mailto:mailautor@xxxx.es}{Nombre/s y Apellido/s del/los Autor/es}\\ \vspace{2mm} %pone el nombre del autor
\scriptsize Organizan:\\
\small \href{http://www.caminos.upm.es/matematicas/jornadas2014/}{\includegraphics[height=12.5mm]{logos_castro_2014.eps}}\\
\scriptsize Centro Internacional de Encuentros Matemáticos\vspace{-2mm}\\ 
\scriptsize Castro Urdiales, España} %xxx-yyy son las páginas inicial y final respectivamente
\date{XX y XX de XXXXX de 2018}
%indispensable que al compilarlo se tenga en el mismo directorio raíz el archivo logomirinconbyn.eps que facilito en la descarga.
\maketitle

\begin{abstract}
En este artículo se pretende dar una visión de \dots{}\dots{} todo un resumen del artículo.\\
\newline
\noindent\textbf{Palabras Clave:} Palabra/s Clave/s 1, Palabra/s Clave/s 2, \dots{}
\newline
\begin{center}
 \textbf{Abstract}
\end{center}

\vspace{1.2mm}A translation of the abstract in english.\\
\newline
\noindent\textbf{Keywords:} Keyword/s 1, Keyword/s 2, \dots{}

\end{abstract}

\section{Título 1}
\subsection{Título 1.1}
\subsubsection{Título 1.1.1.}
Lorem ipsum dolor sit amet, consectetur adipisicing elit, sed do eiusmod tempor incididunt ut labore 
et dolore magna aliqua. Ut enim ad minim veniam, quis nostrud exercitation ullamco laboris nisi ut 
aliquip ex ea commodo consequat. Duis aute irure dolor in reprehenderit in voluptate velit esse cillum 
dolore eu fugiat nulla pariatur. Excepteur sint occaecat cupidatat non proident, sunt in culpa qui 
officia deserunt mollit anim id est laborum. 

%Nuevo párrafo. Para identar el mismo dejar un renglón en blanco de separación entre éste y el anterior párrafo
Lorem ipsum dolor sit amet, consectetur adipisicing elit, sed do eiusmod tempor incididunt ut labore 
et dolore magna aliqua. Ut enim ad minim veniam, quis nostrud exercitation ullamco laboris nisi ut 
aliquip ex ea commodo consequat. Duis aute irure dolor in reprehenderit in voluptate velit esse cillum 
dolore eu fugiat nulla pariatur. Excepteur sint occaecat cupidatat non proident, sunt in culpa qui 
officia deserunt mollit anim id est laborum. 

\begin{quote}
\emph{Aquí irán las citas textuales. En cursiva. Lorem ipsum dolor sit amet, consectetur adipisicing elit, sed do eiusmod tempor incididunt ut labore et dolore magna aliqua. Ut enim ad minim veniam, quis nostrud exercitation ullamco laboris nisi ut 
aliquip ex ea commodo consequat. Duis aute irure dolor in reprehenderit in voluptate velit esse cillum dolore eu fugiat nulla pariatur. Excepteur sint occaecat cupidatat non proident, sunt in culpa qui officia deserunt mollit anim id est laborum.}
\end{quote}

%La siguiente expresión es para las ecuaciones, que irán centradas. Si fuera necesario numerarlas utilizar \begin{equation}\end{equation} 
$$e^x=+\frac{x}{1!}+\frac{x^2}{2!}+\frac{x^3}{3!}+\ldots,\quad -\infty<x<\infty$$

Lorem ipsum dolor sit amet, consectetur adipisicing elit, sed do eiusmod tempor incididunt ut labore 
et dolore magna aliqua. Ut enim ad minim veniam, quis nostrud exercitation ullamco laboris nisi ut 
aliquip ex ea commodo consequat. Duis aute irure dolor in reprehenderit in voluptate velit esse cillum 
dolore eu fugiat nulla pariatur. Excepteur sint occaecat cupidatat non proident, sunt in culpa qui 
officia deserunt mollit anim id est laborum.





\section{Titulo 2}
Lorem ipsum dolor sit amet, consectetur adipisicing elit, sed do eiusmod tempor incididunt ut labore 
et dolore magna aliqua. Ut enim ad minim veniam, quis nostrud exercitation ullamco laboris nisi ut 
aliquip ex ea commodo consequat. Duis aute irure dolor in reprehenderit in voluptate velit esse cillum 
dolore eu fugiat nulla pariatur. Excepteur sint occaecat cupidatat non proident, sunt in culpa qui 
officia deserunt mollit anim id est laborum.

%\begin{comment}
\def\svgwidth{15.05cm}
\vspace*{-4mm}\hspace*{-7mm}\input{lorem.eps_tex}
%\end{comment}

Lorem ipsum dolor sit amet, consectetur adipisicing elit, sed do eiusmod tempor incididunt ut labore 
et dolore magna aliqua. Ut enim ad minim veniam, quis nostrud exercitation ullamco laboris nisi ut 
aliquip ex ea commodo consequat. Duis aute irure dolor in reprehenderit in voluptate velit esse cillum 
dolore eu fugiat nulla pariatur. Excepteur sint occaecat cupidatat non proident, sunt in culpa qui 
officia deserunt mollit anim id est laborum.

\section{Título 3}
%Para figuras a la derecha del párrafo en cuestión. Ver paquete Figure de Latex.
\begin{wrapfigure}{r}{100mm}
\vspace{-4mm}%{-8mm}
    \includegraphics[width=100mm]{logos_castro_2014.eps}
\vspace*{-6mm}\figcaption[]{Pie de Imagen.\footnotemark}\label{figura1}
\vspace*{-3mm}
\end{wrapfigure}
\footnotetext{ \href{http://www.google.es}
%por si es necesario hacer una referencia mediante pie de página a la imagen en cuestión. Si no
%es necesario quitar footnotemark y footnotetext.
{http://www.google.es}}
Lorem ipsum dolor sit amet, consectetur adipisicing elit, sed do eiusmod tempor incididunt ut labore 
et dolore magna aliqua. Ut enim ad minim veniam, quis nostrud exercitation ullamco laboris nisi ut 
aliquip ex ea commodo consequat. Duis aute irure dolor in reprehenderit in voluptate velit esse cillum 
dolore eu fugiat nulla pariatur. Excepteur sint occaecat cupidatat non proident, sunt in culpa qui 
officia deserunt mollit anim id est laborum.

\begin{figure}[htbp!]
 \includegraphics[width=0.7\textwidth]{logos_castro_2014.eps}
\figcaption[]{Pie de Imagen.\footnotemark}
\end{figure}
\footnotetext{ \href{http://www.google.es}
%por si es necesario hacer una referencia mediante pie de página a la imagen en cuestión. Si no
%es necesario quitar footnotemark y footnotetext.
{http://www.google.es}}

Lorem ipsum dolor sit amet, consectetur adipisicing elit, sed do eiusmod tempor incididunt ut labore 
et dolore magna aliqua. Ut enim ad minim veniam, quis nostrud exercitation ullamco laboris nisi ut 
aliquip ex ea commodo consequat. Duis aute irure dolor in reprehenderit in voluptate velit esse cillum 
dolore eu fugiat nulla pariatur. Excepteur sint occaecat cupidatat non proident, sunt in culpa qui 
officia deserunt mollit anim id est laborum.

Lorem ipsum dolor sit amet, consectetur adipisicing elit, sed do eiusmod tempor incididunt ut labore 
et dolore magna aliqua. Ut enim ad minim veniam, quis nostrud exercitation ullamco laboris nisi ut 
aliquip ex ea commodo consequat. Duis aute irure dolor in reprehenderit in voluptate velit esse cillum 
dolore eu fugiat nulla pariatur. Excepteur sint occaecat cupidatat non proident, sunt in culpa qui 
officia deserunt mollit anim id est laborum.

%Ejemplo de Tabla
\begin{table}[htbp!]
\caption{Título de la tabla en parte superior de la misma.\label{patrones}}
 \setlength{\tabcolsep}{0.63mm}
\begin{tabular}{lccccccccccccccccccccl}
\hline
 &  &  & \multicolumn{19}{c}{} Posible longitud de la clave (o factores)  \\ \cline{4-22}
Secuencia & Repetición & Separación & 2 &3&4&5&6&7&8&9&10&11&12&13&14&15&16&17&18&19&20  \\ \hline
HISIEKA & 2 & 112 & \checkmark &  & \checkmark &  &  & \textcolor{red}{\checkmark} & \checkmark &  &  &  &  &  & \checkmark &  & \checkmark &  &  &  &  \\ \hline
    &      & 21  &  & \checkmark &  &  &  & \textcolor{red}{\checkmark} &  &  &  &  &  &  &  &  &  &  &  &  &  \\ \cline{3-22}
    &      & 35    &  &  &  & \checkmark &  & \textcolor{red}{\checkmark} &  &  &  &  &  &  &  &  &  &  &  &  &  \\ \cline{3-22}
    &      & 329    &  &  &  &  &  & \textcolor{red}{\checkmark} &  &  &  &  &  &  &  &  &  &  &  &  &  \\ \cline{3-22}
    &      & 364    & \checkmark &  & \checkmark &  &  & \textcolor{red}{\checkmark} &  &  &  &  &  & \checkmark & \checkmark &  &  &  &  &  &  \\ \cline{3-22}
TIQ & 5    & 385    &  &  &  & \checkmark &  & \textcolor{red}{\checkmark} &  &  &  & \checkmark &  &  &  &  &  &  &  &  &  \\ \cline{3-22}
    &      & 399    &  & \checkmark &  &  &  & \textcolor{red}{\checkmark} &  &  &  &  &  &  &  &  &  &  &  & \checkmark &  \\ \cline{3-22}
    &      & 420    & \checkmark & \checkmark & \checkmark & \checkmark & \checkmark & \textcolor{red}{\checkmark} &  &  & \checkmark &  & \checkmark &  & \checkmark & \checkmark &  &  &  &  & \checkmark \\ \cline{3-22}
    &      & 728    & \checkmark &  & \checkmark &  &  & \textcolor{red}{\checkmark} & \checkmark &  & \checkmark &  &  & \checkmark & \checkmark &  &  &  &  &  &  \\ \cline{3-22}
    &      & 749    &  &  &  &  &  & \textcolor{red}{\checkmark} &  &  &  &  &  &  &  &  &  &  &  &  &  \\ \hline
DRG & 2    & 329 &  &  &  &  &  & \textcolor{red}{\checkmark} &  &  &  &  &  &  &  &  &  &  &  &  &  \\ \hline
VOXSE & 2 & 266 & \checkmark &  &  &  &  & \textcolor{red}{\checkmark} &  &  &  &  &  &  & \checkmark &  &  &  &  & \checkmark &  \\ \hline
\end{tabular}
\end{table}



\begin{thebibliography}{9}


\bibitem{ash}
\textsc{Ash}, M. G.,
\textit{Forced migration and scientific change in the Nazi era}, 
Emigration of Mathematicians and Transmission of Mathematics: Historical Lessons and Consequences of the Third Reich,
Mathematisches Forschungsinstitut Oberwolfach, Nº 51, 2011.

\begin{comment}% Ejemplo de como irá compilada la bibliografía.
\bibitem{butzer}
\textsc{Butzer}, P., \textsc{Volkmann}, L.,
\textit{Otto Blumenthal (1876-1944) in retrospect}, 
Journal of Approximation Theory, Nº 138, pp. 1--36, november 2005.

\bibitem{fernandez}
\textsc{Fernández}, S.,
\textit{La Criptografía Clásica}, Revista SIGMA, Nº 24, pp. 119--141, abril 2004.

\bibitem{garron}
\textsc{Garron}, L., 
\textit{National Socialism and the Death of German Mathematics}, 
Stanford-in-Berlin Program course ``Science, Medicine, and Technology in Nazi Germany'',
december 2010.

\bibitem{hodges}
\textsc{Hodges}, A., 
\textit{The Military Use of Alan Turing}, 
Mathematics and War, pp. 312--325, Bernhelm Booss Bavnbek and Jens H\o{}yrup Editors, Birkhäuser,
2003.

\bibitem{lahoz}
\textsc{Lahoz-Beltrá}, R., 
\textit{Turing: Del primer ordenador a la inteligencia artificial}, 
Colección: La matemática y sus personajes, Nº 24, 1ª Edición. Nívola, Madrid, 2005.

\bibitem{maclane}
\textsc{MacLane}, S., 
\textit{Mathematics at Göttingen under the Nazis},
Notices of AMS, Vol. 42, Nº 10, october 1995.

\bibitem{miller}
\textsc{Miller}, A. R., 
\textit{The Criptographic Mathematics of Enigma},
Center for Crypologic History, 1996.

\bibitem{oberdiek}
\textsc{Oberdiek}, A., 
\textit{Göttinger Universitäts-Bauten. Die Baugeschichte der
Georg-August-Univeristät}, Verlag Göttinger Tageblatt \textsc{GMBH \& CO. KG}, p. 46, 1989.

\bibitem{remmert}
\textsc{Remmert}, V. R., 
\textit{Mathematical Publishing in the Third Reich: Springer-Verlag and the Deutsche Mathematiker-Vereinigung},
The Mathematical Intelligencer, Springer-Verlag, New York, 2000.

\bibitem{segal}
\textsc{Segal}, S. L., 
\textit{Mathematicians under Nazis},
Princeton University Press, 2003.

\bibitem{segal2}
------,
\textit{Mathematics and German Politics: The National Socialist Experience},
Historia Mathematica, Nº 13, pp. 118--135, 1986.

\bibitem{siegmund}
\textsc{Siegmund-Schultze}, R., 
\textit{Mathematicians fleeing from Nazi Germany. Individual Fates and Global Impact},
Princeton University Press, 2009.

\bibitem{siegmund2}
------,
\textit{Emigration of mathematicians and of mathematics: facts and open questions}, 
Emigration of Mathematicians and Transmission of Mathematics: Historical Lessons and Consequences of the Third Reich,
Mathematisches Forschungsinstitut Oberwolfach, Nº 51, 2011.

\bibitem{singh}
\textsc{Singh}, S., 
\textit{Los Códigos Secretos: El arte y la ciencia de la criptografía, desde el antiguo Egipto a la era Internet}, Editorial Debate, 2000.  

\bibitem{smith}
\textsc{Smith}, D., \textsc{Simmons}, C., 
\textit{The Effect of the Nazi Regime on the World of Mathematics and Individual Mathematicians},
University of Central Oklahoma, 2010.

\bibitem{tobies}
\textsc{Tobies}, R.,
\textit{Expelled female mathematicians in exile: working conditions, and the impact on pure and applied mathematics}, 
Emigration of Mathematicians and Transmission of Mathematics: Historical Lessons and Consequences of the Third Reich,
Mathematisches Forschungsinstitut Oberwolfach, Nº 51, 2011.
\end{comment}

\end{thebibliography}



\vspace{1cm}
\noindent\textbf{Sobre el/los autor/es:}\\
\emph{Nombre:} Nombre de autor 1\\
\emph{Correo electrónico:} \href{mailto:autor1@xxxxxx.es}{autor1@xxxxxx.es}\\
\emph{Institución:} A la que pertenece (instituto, universidad, etc).

\noindent\emph{Nombre:} Nombre de autor 2\\
\emph{Correo electrónico:} \href{mailto:autor2@xxxxxx.es}{autor2@xxxxxx.es}\\
\emph{Institución:} A la que pertenece (instituto, universidad, etc).\\



\end{document}